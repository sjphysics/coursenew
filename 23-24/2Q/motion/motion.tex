\documentclass[12pt,addpoints]{exam}
%\usepackage{enumitem}
\usepackage{amsfonts,amssymb,amsmath, amsthm}
\usepackage{graphicx}
\usepackage{systeme}
\usepackage{pgf,tikz,pgfplots}
\pgfplotsset{compat=1.15}
\usepgfplotslibrary{fillbetween}
\usepackage{mathrsfs}
\usetikzlibrary{arrows}
\usetikzlibrary{calc}
\usepackage{geometry}
\geometry{
	a4paper,
	total={170mm,257mm},
	left=20mm,
	top=7mm,
}
\pagestyle{headandfoot}
%\firstpageheadrule
\runningheader{Motion}{}{Page \thepage\ of \numpages}
\runningheadrule
\firstpagefooter{}{}{}
\runningfooter{By Aaron G.K.}{}{Page \thepage\ of \numpages}
\title{Motion}
\author{Aaron G.K.}
\begin{document}
	\maketitle
	\begin{center}
		\subsection*{A Straight Line Motion}
	\end{center}
	There is nothing more fundamental in the study of physics than motion. We will bring a lot of mathematics to discuss this subject, but we are going to start as simply as possible – with motion of a single particle that remains on a straight line. This simplifies our task in more ways than one:
	\begin{enumerate}
		\item By restricting ourselves to single particles, we don't have to worry about the complicated motions of systems of particles, where each of the particles can move differently than the others.
		\item By keeping the motion along a straight line, there are only two directions involved, and these directions can be characterized simply as "positive" and "negative" – there is no need for unit vectors.
	\end{enumerate}
	\begin{center}
		\textbf{Displacement}
	\end{center}
	In order for motion to occur for a body, its position obviously must change from one instant in time to another. We will refer to the coordinate position of the straight line on which the object moves as $x(t)$. A change in this position we call the displacement, and refer to it as a change in position:
	$$\text{displacement} = \Delta x \equiv x_f - x_i$$
	Notice that if the final position is a smaller number than the initial position, then the object has a negative displacement. Eventually we will treat displacement as a vector, but for our straight-line motion, the sign of the value provides all the information we need about the direction. 
 	\begin{center}\textbf{Average Velocity}	 \end{center}
	Thankfully, there is more to motion than just displacement. We will generally also be interested in how fast that displacement occurs. We therefore define a \textbf{rate} called the average velocity thus:
	$$\text{average velocity} = v_{ave} \equiv \dfrac{\Delta x}{\Delta t} = \dfrac{x_f-x_i}{t_f-t_i}$$
	Since we know displacement is a vector (\textit{and in our current simple 1-dimensional model it can only have two directions}), then average velocity must be a vector as well.
	\begin{center}
		\textbf{Instantaneous Velocity}
	\end{center}
	It is usually not informative enough to just talk about the starts and the ends? That is, how do we define a velocity at a single moment in time – the instantaneous velocity? Well, we know the answer to this from elementary calculus. We start with our idea of average velocity, and just shrink the time span down very small, until it vanishes:
	$$\text{instantaneous velocity} = v = \lim_{\Delta t \rightarrow 0} \dfrac{\Delta x}{\Delta t} = \dfrac{dx}{dt}$$
	\begin{center}
		\textbf{Average and Instantaneous Acceleration}
	\end{center}
	In simple rectilinear motions, we take our discussion of motion to one level more – we consider that things might speed up or slow down. Just as we defined average velocity in terms of before and after positions, we also define average acceleration in terms of before and after (instantaneous) velocities:
	$$\text{average acceleration} = a_{ave} = \dfrac{\Delta v}{\Delta t} = \dfrac{v_f-v_o}{t_f-t_o}$$
	And, as before, we use calculus to extend this notion of average acceleration to instantaneous acceleration, which we describe as the amount that our object is speeding up or slowing down at a single moment in time:
	\begin{align} \text{instantaneous acceleration} = a = \lim_{\Delta t \rightarrow 0} \dfrac{\Delta v}{\Delta t} &= \dfrac{dv}{dt} \\[5pt] &= \dfrac{d^2x}{dt^2}\end{align}
	Now that we have the basics covered, let's go over the rest
	
	

\end{document}	