\hypertarget{position-displacement-and-average-velocity}{%
\subsection{3.1 Position, Displacement, and Average
Velocity}\label{position-displacement-and-average-velocity}}

\hypertarget{learning-objectives}{%
\subsubsection{Learning Objectives}\label{learning-objectives}}

By the end of this section, you will be able to:

\begin{itemize}
\item
  Define position, displacement, and distance traveled.
\item
  Calculate the total displacement given the position as a function of
  time.
\item
  Determine the total distance traveled.
\item
  Calculate the average velocity given the displacement and elapsed
  time.
\end{itemize}

When you're in motion, the basic questions to ask are: Where are you?
Where are you going? How fast are you getting there? The answers to
these questions require that you specify your position, your
displacement, and your average velocity---the terms we define in this
section.

\hypertarget{position}{%
\subsubsection{Position}\label{position}}

To describe the motion of an object, you must first be able to describe
its position (\emph{x}): \emph{where it is at any particular time}. More
precisely, we need to specify its position relative to a convenient
frame of reference. A frame of reference is an arbitrary set of axes
from which the position and motion of an object are described. Earth is
often used as a frame of reference, and we often describe the position
of an object as it relates to stationary objects on Earth. For example,
a rocket launch could be described in terms of the position of the
rocket with respect to Earth as a whole, whereas a cyclist's position
could be described in terms of where she is in relation to the buildings
she passes \protect\hyperlink{CNX_UPhysics_03_01_Cyclists}{Figure 3.2}.
In other cases, we use reference frames that are not stationary but are
in motion relative to Earth. To describe the position of a person in an
airplane, for example, we use the airplane, not Earth, as the reference
frame. To describe the position of an object undergoing one-dimensional
motion, we often use the variable \emph{x}. Later in the chapter, during
the discussion of free fall, we use the variable \emph{y}.

\includegraphics[width=3.24667in,height=2.44in]{media/rId24.jpeg}

Figure 3.2 These cyclists in Vietnam can be described by their position
relative to buildings or a canal. Their motion can be described by their
change in position, or displacement, in a frame of reference. (credit:
modification of work by Suzan Black)

\hypertarget{displacement}{%
\subsubsection{Displacement}\label{displacement}}

If an object moves relative to a frame of reference---for example, if a
professor moves to the right relative to a whiteboard
\protect\hyperlink{CNX_UPhysics_03_01_Chalkboard}{Figure 3.3}---then the
object's position changes. This change in position is called
displacement. The word \emph{displacement} implies that an object has
moved, or has been displaced. Although position is the numerical value
of \emph{x} along a straight line where an object might be located,
displacement gives the \emph{change} in position along this line. Since
displacement indicates direction, it is a vector and can be either
positive or negative, depending on the choice of positive direction.
Also, an analysis of motion can have many displacements embedded in it.
If right is positive and an object moves 2 m to the right, then 4 m to
the left, the individual displacements are 2 m and \(- 4\) m,
respectively.

\includegraphics[width=4.59333in,height=3.62667in]{media/rId29.jpeg}

Figure 3.3 A professor paces left and right while lecturing. Her
position relative to Earth is given by \emph{x}. The +2.0-m displacement
of the professor relative to Earth is represented by an arrow pointing
to the right.

\hypertarget{displacement-1}{%
\subsubsection{Displacement}\label{displacement-1}}

Displacement \(\mathrm{\Delta}x\) is the change in position of an
object:

\(\mathrm{\Delta}x = x_{\mathrm{f}} - x_{0},\)

3.1

where \(\mathrm{\Delta}x\) is displacement, \(x_{\mathrm{f}}\) is the
final position, and \(x_{0}\) is the initial position.

We use the uppercase Greek letter delta (Δ) to mean ``change in''
whatever quantity follows it; thus, \(\mathrm{\Delta}x\) means
\emph{change in position} (final position less initial position). We
always solve for displacement by subtracting initial position \(x_{0}\)
from final position \(x_{\mathrm{f}}\). Note that the SI unit for
displacement is the meter, but sometimes we use kilometers or other
units of length. Keep in mind that when units other than meters are used
in a problem, you may need to convert them to meters to complete the
calculation (see
\href{http://openstax.org/books/university-physics-volume-1/pages/b-conversion-factors}{Appendix
B}).

Objects in motion can also have a series of displacements. In the
previous example of the pacing professor, the individual displacements
are 2 m and \(- 4\) m, giving a total displacement of −2 m. We define
total displacement \(\mathrm{\Delta}x_{\mathrm{\text{Total}}}\), as
\emph{the sum of the individual displacements}, and express this
mathematically with the equation

\(\mathrm{\Delta}x_{\mathrm{\text{Total}}} = \sum\mathrm{\Delta}x_{\mathrm{i}},\)

3.2

where \(\mathrm{\Delta}x_{i}\) are the individual displacements. In the
earlier example,

\(\mathrm{\Delta}x_{1} = x_{1} - x_{0} = 2 - 0 = 2\ \mathrm{\text{m.}}\)

Similarly,

\(\mathrm{\Delta}x_{2} = x_{2} - x_{1} = - 2 - (2) = - 4\ \mathrm{\text{m.}}\)

Thus,

\(\mathrm{\Delta}x_{\mathrm{\text{Total}}} = \mathrm{\Delta}x_{1} + \mathrm{\Delta}x_{2} = 2 - 4 = - 2\ \mathrm{m}\mathrm{.}\)

The total displacement is 2 − 4 = −2 m along the \emph{x}-axis. It is
also useful to calculate the magnitude of the displacement, or its size.
The magnitude of the displacement is always positive. This is the
absolute value of the displacement, because displacement is a vector and
cannot have a negative value of magnitude. In our example, the magnitude
of the total displacement is 2 m, whereas the magnitudes of the
individual displacements are 2 m and 4 m.

The magnitude of the total displacement should not be confused with the
distance traveled. Distance traveled \(x_{\mathrm{\text{Total}}}\), is
the total length of the path traveled between two positions. In the
previous problem, the distance traveled is the sum of the magnitudes of
the individual displacements:

\(x_{\mathrm{\text{Total}}} = \left| \mathrm{\Delta}x_{1} \right| + \left| \mathrm{\Delta}x_{2} \right| = 2 + 4 = 6\ \mathrm{m}\mathrm{.}\)

\hypertarget{average-velocity}{%
\subsubsection{Average Velocity}\label{average-velocity}}

To calculate the other physical quantities in kinematics we must
introduce the time variable. The time variable allows us not only to
state where the object is (its position) during its motion, but also how
fast it is moving. How fast an object is moving is given by the rate at
which the position changes with time.

For each position \(x_{\mathrm{i}}\), we assign a particular time
\(t_{\mathrm{i}}\). If the details of the motion at each instant are not
important, the rate is usually expressed as the average velocity
\(\overset{\mathrm{–}}{v}\). This vector quantity is simply the total
displacement between two points divided by the time taken to travel
between them. The time taken to travel between two points is called the
elapsed time \(\mathrm{\Delta}t\).

\hypertarget{average-velocity-1}{%
\subsubsection{Average Velocity}\label{average-velocity-1}}

If \(x_{1}\) and \(x_{2}\) are the positions of an object at times
\(t_{1}\) and \(t_{2}\), respectively, then

\includegraphics[width=3.64583in,height=0.66667in]{media/rId48.png}

3.3

It is important to note that the average velocity is a vector and can be
negative, depending on positions \(x_{1}\) and \(x_{2}\).

\hypertarget{example-3.1}{%
\subsubsection{Example 3.1 }\label{example-3.1}}

\hypertarget{delivering-flyers}{%
\paragraph{Delivering Flyers}\label{delivering-flyers}}

Jill sets out from her home to deliver flyers for her yard sale,
traveling due east along her street lined with houses. At \(0.5\) km and
9 minutes later she runs out of flyers and has to retrace her steps back
to her house to get more. This takes an additional 9 minutes. After
picking up more flyers, she sets out again on the same path, continuing
where she left off, and ends up 1.0 km from her house. This third leg of
her trip takes \(15\) minutes. At this point she turns back toward her
house, heading west. After \(1.75\) km and \(25\) minutes she stops to
rest.

\begin{enumerate}
\def\labelenumi{\alph{enumi}.}
\item
  What is Jill's total displacement to the point where she stops to
  rest?
\item
  What is the magnitude of the final displacement?
\item
  What is the average velocity during her entire trip?
\item
  What is the total distance traveled?
\item
  Make a graph of position versus time.
\end{enumerate}

A sketch of Jill's movements is shown in
\protect\hyperlink{CNX_UPhysics_03_01_Flyers}{Figure 3.4}.

\includegraphics[width=5.23333in,height=2.22in]{media/rId51.jpeg}

Figure 3.4 Timeline of Jill's movements.

\hypertarget{strategy}{%
\paragraph{Strategy}\label{strategy}}

The problem contains data on the various legs of Jill's trip, so it
would be useful to make a table of the physical quantities. We are given
position and time in the wording of the problem so we can calculate the
displacements and the elapsed time. We take east to be the positive
direction. From this information we can find the total displacement and
average velocity. Jill's home is the starting point \(x_{0}\). The
following table gives Jill's time and position in the first two columns,
and the displacements are calculated in the third column.

\begin{longtable}[]{@{}lll@{}}
\toprule
Time \emph{t}\textsubscript{i} (min) & Position \(x_{i}\) (km) &
Displacement \(\mathrm{\Delta}x_{\mathrm{i}}\) (km)\tabularnewline
\midrule
\endhead
\(t_{0} = 0\) & \(x_{0} = 0\) &
\(\mathrm{\Delta}x_{0} = 0\)\tabularnewline
\(t_{1} = 9\) & \(x_{1} = 0.5\) &
\(\mathrm{\Delta}x_{1} = x_{1} - x_{0} = 0.5\)\tabularnewline
\(t_{2} = 18\) & \(x_{2} = 0\) &
\(\mathrm{\Delta}x_{2} = x_{2} - x_{1} = - 0.5\)\tabularnewline
\(t_{3} = 33\) & \(x_{3} = 1.0\) &
\(\mathrm{\Delta}x_{3} = x_{3} - x_{2} = 1.0\)\tabularnewline
\(t_{4} = 58\) & \(x_{4} = - 0.75\) &
\(\mathrm{\Delta}x_{4} = x_{4} - x_{3} = - 1.75\)\tabularnewline
\bottomrule
\end{longtable}

\hypertarget{solution}{%
\paragraph{Solution}\label{solution}}

\begin{enumerate}
\def\labelenumi{\alph{enumi}.}
\item
  From the above table, the total displacement is
\end{enumerate}

\begin{itemize}
\item
  \(\sum\mathrm{\Delta}x_{\mathrm{i}} = 0.5 - 0.5 + 1.0 - 1.75\ \mathrm{\text{km}} = - 0.75\ \mathrm{\text{km}}\mathrm{.}\)
\end{itemize}

\begin{enumerate}
\def\labelenumi{\alph{enumi}.}
\setcounter{enumi}{1}
\item
  The magnitude of the total displacement is
  \(\left| - 0.75 \right|\ \mathrm{\text{km}} = 0.75\ \mathrm{\text{km}}\).
\item
  \(\mathrm{\text{Average\ velocity}} = \frac{\mathrm{\text{Total}}\ \mathrm{\text{displacement}}}{\mathrm{\text{Elapsed}}\ \mathrm{\text{time}}} = \overset{\mathrm{–}}{v} = \frac{- 0.75\ \mathrm{\text{km}}}{58\ \mathrm{\min}} = - 0.013\ \mathrm{km/min}\)
\item
  The total distance traveled (sum of magnitudes of individual
  displacements) is
  \(x_{\mathrm{\text{Total}}} = \sum\left| \mathrm{\Delta}x_{\mathrm{i}} \right| = 0.5 + 0.5 + 1.0 + 1.75\ \mathrm{\text{km}} = 3.75\ \mathrm{\text{km}}\).
\item
  We can graph Jill's position versus time as a useful aid to see the
  motion; the graph is shown in
  \protect\hyperlink{CNX_UPhysics_03_01_Pos}{Figure 3.5}.
\end{enumerate}

\begin{itemize}
\item
  \includegraphics[width=2.75333in,height=2.63333in]{media/rId54.jpeg}

  Figure 3.5 This graph depicts Jill's position versus time. The average
  velocity is the slope of a line connecting the initial and final
  points.
\end{itemize}

\hypertarget{significance}{%
\paragraph{Significance}\label{significance}}

Jill's total displacement is −0.75 km, which means at the end of her
trip she ends up \(0.75\ \mathrm{\text{km}}\) due west of her home. The
average velocity means if someone was to walk due west at \(0.013\)
km/min starting at the same time Jill left her home, they both would
arrive at the final stopping point at the same time. Note that if Jill
were to end her trip at her house, her total displacement would be zero,
as well as her average velocity. The total distance traveled during the
58 minutes of elapsed time for her trip is 3.75 km.

~

\hypertarget{check-your-understanding-3.1}{%
\subsubsection{Check Your Understanding 3.1
}\label{check-your-understanding-3.1}}

A cyclist rides 3 km west and then turns around and rides 2 km east. (a)
What is his displacement? (b) What is the distance traveled? (c) What is
the magnitude of his displacement?

\includegraphics[width=4.54in,height=1.37333in]{media/rId57.jpeg}
