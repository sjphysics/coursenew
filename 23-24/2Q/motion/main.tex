\documentclass[12pt,addpoints]{exam}
%\usepackage{enumitem}
\usepackage{amsfonts,amssymb,amsmath, amsthm}
\usepackage{graphicx}
\usepackage{systeme}
\usepackage{pgf,tikz,pgfplots}
\pgfplotsset{compat=1.15}
\usepgfplotslibrary{fillbetween}
\usepackage{mathrsfs}
\usetikzlibrary{arrows}
\usetikzlibrary{calc}
\usepackage{geometry}
\geometry{
	a4paper,
	total={170mm,257mm},
	left=20mm,
	top=15mm,
}
\pagestyle{headandfoot}
%\firstpageheadrule
\runningheader{Motion}{}{Page \thepage\ of \numpages}
\runningheadrule
\firstpagefooter{}{}{}
\runningfooter{By Aaron G.K.}{}{Page \thepage\ of \numpages}
\title{Motion}
\author{Aaron G.K.}
\begin{document}
	\maketitle
	\begin{center}
		\subsection*{A Straight Line Motion}
	\end{center}
	There is nothing more fundamental in the study of physics than motion. We will bring a lot of mathematics to discuss this subject, but we are going to start as simply as possible – with motion of a single particle that remains on a straight line. This simplifies our task in more ways than one:
	\begin{enumerate}
		\item By restricting ourselves to single particles, we don't have to worry about the complicated motions of systems of particles, where each of the particles can move differently than the others.
		\item By keeping the motion along a straight line, there are only two directions involved, and these directions can be characterized simply as "positive" and "negative" – there is no need for unit vectors.
	\end{enumerate}
	\begin{center}
		\textbf{Displacement}
	\end{center}
	In order for motion to occur for a body, its position obviously must change from one instant in time to another. We will refer to the coordinate position of the straight line on which the object moves as $x(t)$. A change in this position we call the displacement, and refer to it as a change in position:
	$$\text{displacement} = \Delta x \equiv x_f - x_i$$
	Notice that if the final position is a smaller number than the initial position, then the object has a negative displacement. Eventually we will treat displacement as a vector, but for our straight-line motion, the sign of the value provides all the information we need about the direction. 
 	\begin{center}\textbf{Average Velocity}	 \end{center}
	Thankfully, there is more to motion than just displacement. We will generally also be interested in how fast that displacement occurs. We therefore define a \textbf{rate} called the average velocity thus:
	$$\text{average velocity} = v_{ave} \equiv \dfrac{\Delta x}{\Delta t} = \dfrac{x_f-x_i}{t_f-t_i}$$
	Since we know displacement is a vector (\textit{and in our current simple 1-dimensional model it can only have two directions}), then average velocity must be a vector as well.
	\begin{center}
		\textbf{Instantaneous Velocity}
	\end{center}
	It is usually not informative enough to just talk about the starts and the ends? That is, how do we define a velocity at a single moment in time – the instantaneous velocity? Well, we know the answer to this from elementary calculus. We start with our idea of average velocity, and just shrink the time span down very small, until it vanishes:
	$$\text{instantaneous velocity} = v = \lim_{\Delta t \rightarrow 0} \dfrac{\Delta x}{\Delta t} = \dfrac{dx}{dt}$$
	\begin{center}
		\textbf{Average and Instantaneous Acceleration}
	\end{center}
	In simple rectilinear motions, we take our discussion of motion to one level more – we consider that things might speed up or slow down. Just as we defined average velocity in terms of before and after positions, we also define average acceleration in terms of before and after (instantaneous) velocities:
	$$\text{average acceleration} = a_{ave} = \dfrac{\Delta v}{\Delta t} = \dfrac{v_f-v_o}{t_f-t_o}$$
	And, as before, we use calculus to extend this notion of average acceleration to instantaneous acceleration, which we describe as the amount that our object is speeding up or slowing down at a single moment in time:
	\begin{align} \text{instantaneous acceleration} = a = \lim_{\Delta t \rightarrow 0} \dfrac{\Delta v}{\Delta t} &= \dfrac{dv}{dt} \\[5pt] &= \dfrac{d^2x}{dt^2}\end{align}
	Now that we have the basics covered, let's go over some of the important analytical parts.
	\begin{center}
		\subsection*{Equations of Uniformly Accelerated Motion}
	\end{center}
	It is important to note that we are not yet dealing with causes for these motions, but only the motions themselves, hence, we call the topic of our current study \textit{kinematics}. We will mostly only deal with constant accelerations, and since instantaneous acceleration is the derivative of velocity, it is not difficult to integrate(\textit{compute the anti-derivative}) it to get the instantaneous velocity as a function of time:
	$$\left. \begin{array}{l} a = \dfrac{dv}{dt}\;\;\; \Rightarrow \;\;\;v\left( t \right) = \int {a\;dt} = at + const \;\\ const = v\left( 0 \right) \equiv {v_i} \end{array} \right\}\;\;\;v\left( t \right) = at + {v_i}$$
	The constant of integration is found by plugging  $t=0$
	into the above equation, which results in the velocity of the object at the starting time, which is typically designated as $v_i$. \\ \\
	We can perform the exact steps again to obtain the equation of motion for position as a function of time, since we know how it relates to the instantaneous velocity:
	$$\left. \begin{array}{l} v = \dfrac{dx}{dt}\;\;\; \Rightarrow \;\;\;x\left( t \right) = \int {vdt} = \int {\left( {at + {v_i}} \right)dt} = \frac{1}{2}a{t^2} + {v_i}t + const \;\\ const = x\left( 0 \right) \equiv {x_i} \end{array} \right\}\;\;\;x\left( t \right) = \frac{1}{2}a{t^2} + {v_i}t + {x_i}$$
	Bear in mind that if we have all the details of this last equation, we can obtain the velocity equation above simply by taking a derivative. We cannot go in the opposite direction without also obtaining the starting position. \\ \\
	Let’s take a look at the common numbers we can encounter in a constant-acceleration situation:
	\begin{itemize}
		\item independent variable: $t$
		\item dependent variables:  $x$, $v$
		\item constants of the motion:  $x_i$, $v_i$, $a$ (acceleration is constant by assumption)
	\end{itemize}
	The most common useful re-combining of these variables involves eliminating time from the two equations, since you may be given velocities and positions. The algebra is straightforward:
	$$\left. \begin{array}{l} v_f = at + v_i \;\Rightarrow \;t = \dfrac{v_f - v_i}{a}\\ x_f - x_i = \frac{1}{2}a{t^2} + {v_i}t \end{array} \right\} \;\;\; x_f - x_i = \frac{1}{2}a \left( \dfrac{v_f - v_i}{a} \right)^2 + v_i \left( \dfrac{v_f - v_i}{a} \right) \;\;\;\Rightarrow \;\;\; 2a \left( x_f - x_i \right) = {v_f}^2 - {v_i}^2$$
	You can think of this equation as the “before/after” equation, because it deals only with starting and ending positions and velocities, and has eliminated time as an input variable.
	$${v_{ave}} = \dfrac{x_f - x_i}{t} = \dfrac{x \left( t \right) - x_i}{t}\;\;\;\Rightarrow \;\;\; x \left( t \right) = v_{ave}t + x_i$$
	$${v_{ave}} = \dfrac{x_f - x_i}{t} = \dfrac{\frac{1}{2}a{t^2} + {v_i}t}{t} = \frac{1}{2}at + v_i = \frac{1}{2}\left( v_f - v_i \right) + v_i \;\;\; \Rightarrow \;\;\; v_{ave} = \dfrac{v_i + v_f}{2}$$
	$$x \left( t \right) = (\dfrac{v_i + v_f}{2})t + x_i$$
	\begin{center}
		\subsection*{Free-Fall}
	\end{center}
	Assuming air resistance has a small effect (remember, we are devising a simplified model here), then it turns out (as shown by Galileo dropping stones from the Tower of Pisa, and more dramatically in the demonstration) that objects all accelerate at the same constant rate as they fall to Earth. This rate of acceleration is commonly given the symbol $g$, and it has the value:
	$$\text{acceleration due to gravity near the surface of the earth}=g=9.8m/s^2$$
	This acceleration is of course always directed downward, and depending on our choice of coordinate system \& reference frames, this can be either positive or negative, that is $a=\pm g$. Once the coordinate system is selected, the sign for $a$ stays the same no matter which way the object is moving. \\ \\
	If the positive direction is chosen to be upward, and the object is moving upward, then its velocity is positive and the negative value of $a$
	leads to a slowing of the object’s motion. If it is moving down, then its velocity is negative, and the negative acceleration leads to the velocity becoming more negative (i.e. it is speeding up).
	\begin{center}
		\textbf{Graphs of Motion}
	\end{center}
    We conclude our discussion of straight-line motion by taking on the topic of representing motion with graphs. These graphs represent what is happening to the various dependent variables ($x$, $v$, and $a$) over time. There are three goals here:
    \begin{enumerate}
    	\item To interpret a graph in terms of the physical motion of the object it represents.
    	\item To sketch a graph that represents the physical motion of an object, given a description of that motion.
    	\item To sketch a graph of one or two dependent variables based on the graph of another dependent variable.
    \end{enumerate}
    These are not always easy tasks to perform, for two main reasons: First, our first instinct when we see a graph is to interpret it as a picture, rather than a plot of a quantity vs. time. The second problem (\textit{and this persists throughout the study of physics}) is the tendency to confuse the change of a quantity for the value of that quantity. More precisely, we tend to lose sight of the fact that a variable's value at an instant and its rate of change are quite independent of each other. \\ \\
    Here are some of the few important points we need to understand during while interpreting and plotting graphs.
    
    
    
\end{document}	