\documentclass[12pt,addpoints]{exam}
\usepackage{enumitem}
\usepackage{amsfonts,amssymb,amsmath, amsthm}
\usepackage{graphicx}
\usepackage{systeme}
\usepackage{pgf,tikz,pgfplots}
\pgfplotsset{compat=1.15}
\usepgfplotslibrary{fillbetween}
\usepackage{mathrsfs}
\usetikzlibrary{arrows}
\usetikzlibrary{calc}\author{St John Baptist De La Salle Catholic School, Addis Ababa}
\usepackage{geometry}
\date{October, 2023}
\geometry{
	a4paper,
	total={170mm,257mm},
	left=15mm,
	right=15mm,
	bottom=20mm,
	top=5mm,
}
\begin{document}
	\title{Grade 11 Physics Workbook}
	\maketitle
	
	\begin{center}
		\section*{Physics \& Human Society}
		\subsection*{Chapter Summary}
		\textbf{Physics, a Beautiful Science} 
	\end{center}
		
	\begin{itemize}
		\item Physics is a natural science that involves the study of matter and its motion through space and time, along with related concepts such as energy and force. It is the study of the universe itself; how it began, how it operates, and how it will be done.
		\item Many scientific disciplines, such as biophysics, physical chemistry and engineering, are hybrids of physics and other sciences.
		\item The application of physics is fundamental towards significant contributions in new technologies that arise from theoretical breakthroughs that we use daily. All electronics, cryptographic encryptions, medical instruments, the internet itself, and more have been the direct result of physics.
		\begin{center}
			\textbf{Models, Theories \& Laws} 
		\end{center}
		\item Concepts in physics cannot be proven, they can only be supported or disproven through observation and experimentation. 
		\item A model is an evidence-based representation of something that is either too difficult or impossible to display directly.
		\item A theory is an explanation for patterns in nature that is supported by scientific evidence and verified multiple times by various groups of researchers.
		\item A law uses concise language, often expressed as a mathematical equation, to describe a generalized pattern in nature that is supported by scientific evidence and repeated experiments.
		\item A background in physics is probably one of the most versatile career paths. You can use your skills to work in multiple disciplines including but not limited to  astronomy, healthcare, engineering, energy, technology, meteorology, finance, and the like..
		\item Recent developments in physics include discovery and imaging of black holes, quantum cryptography, quantum computing, AI \& ML, Gravitational Waves, Understanding of the Universe through many missions like the recent JWST, Neutrino Astronomy, and many more exciting stuff. 
		\begin{center}
		\textbf{The Scientific Method} 	
		\end{center}
		The scientific method provides scientists with a well structured
		scientific platform to help find the answers to their questions. It can be summarized as follows \begin{enumerate}
			\item Ask a question about the world around you.
			\item Do background research on your questions.
			\item Make a hypothesis about the event that gives a sensible result. You must be able to test your hypothesis through experiment.
			\item Design an experiment to test the hypothesis. These methods must be repeatable
			and follow a logical approach.
			\item Collect data accurately and interpret the data. You must be able to take measurements, collect information, and present your data in a useful format (drawings,
			explanations, tables and graphs).
			\item Draw conclusions from the results of the experiment. Your observations must be made objectively, never force the data to fit your hypothesis.
			\item Decide whether your hypothesis explains the data collected accurately.
			\item If the data fits your hypothesis, verify your results by repeating the experiment or getting someone else to repeat the experiment.
			\item If your data does not fit your hypothesis perform more background research a make a new hypothesis.
		\end{enumerate}
	\end{itemize}
	\begin{center}
	\subsection*{Questions}	
	\end{center}
	\begin{questions}
		\question How is a model different from a theory?(For instance, you have learned phrases like "the Dalton Atomic Model" and "Kinetic Theory of Gases".. why is one a model and the other a theory?)\vspace{2in}
		\question What is the difference between algorithms that are \textit{computationally secure} and those that are \textit{information theoretically secure}?\vspace{2in}
		\question What are common ways people's passwords and personal information get compromised? How can one protect themselves from such attacks?\vspace{2in}
		\question We have discussed in class the discrepancy between the same observation but different analyses of a data set. We took the news regarding observations by JWST data being interpreted differently by different groups of scientists. A prominent bloc suggest that the electromagnetically undetectable, but seemingly gravitationally present "matter" is dark matter while dissidents suggest that Newton's Universal Gravitation Law applies differently for distant large galaxies. Which idea do you seem to agree more with? Why? \vspace{2in}
		\question Research is extremely important for many different reasons; we recently saw its significance during the COVID-19 pandemic when many otherwise expensive journals were pressured into making the pandemic related researches published and accessible for free. The free movement of research led to many people believing in the methodology of the vaccine production and even more others chipping in ideas to better the research being done. Now that the pandemic is over, many of those articles are back on paywall and one has to pay large sums to access the research articles. Explain whether you agree on the current model of article publications or whether you think research articles should be open access always. Give examples of open access journals and/or research repositories. \vspace{3in}
		\question In collaborations like the CERN (European Organization for Nuclear Research), there are multiple people from different backgrounds, institutions, and even countries collaborating to do research. CERN is the home of many experiments such as ALICE(A Large Ion Collider Experiment), LHC(the Large Hadron Collider), CMS(Compact Muon Solenoid) which have been significant in many high energy physics experiments such as the one done to discover the famous Higgs Boson.
		\begin{itemize}
			\item What (simply )is the significance of the Higgs Boson? \vspace{0.5in}
			\item Why is such a big collaboration important in your opinion? Why wasn't, say the Higgs Boson discovered in a small lab of 4 people in a rural town? \\ \vspace{2in}
			\item CMS(Compact Muon Solenoid) is a general purpose particle accelerator. Say, for example, a theorist proposed existence of a particle that you would like to detect. Based on electromagnetism concepts we discussed last semester, how would you design an experiment to detect said particle?\vspace{2in}
		\end{itemize}
		\question Physics communities are important to nourish underrepresented members of the community and also generally to advocate for physics within the general community. Out of the communities we have seen in class, which ones do you feel most interested by? Why? \vspace{1in}
		\question Your hobbies are probably directly or indirectly related to physics. List some of your hobbies and explain how physics plays a role. \vspace{2in}
		\question What is the difference between experiential and non-experiential knowledge?\vspace{1in}
		\question It is true that in science(as in any other field), rivalries do exist. It wouldn't be an exaggeration to, perhaps, suggest that scientific rivalries may even be on the fiercer side of the rivalry spectrum, so to speak. Some of these rivalries include ones between \textit{Nikola Tesla} and \textit{Thomas Edison}, \textit{Isaac Newton} and \textit{Gottfried Leibniz}, \textit{Isaac Newton} and \textit{Robert Hooke} and ones between countries. It would be wrong to suggest that rivalries serve no purpose, however, in the pursuit of scientific truth, working with adversaries can pay off. A research  dubbed \textit{Adversarial Collaboration Project} which ran at UPenn hypothesized that for advancement of scientific facts, rivals would have to work together. What is your take on this? How different do you think the world could have been if adversarial collaborations were a common practice since the early days of science?\vspace{3in}
		\begin{center}
			\section*{Vectors}
			\subsection*{Chapter Summary}
			\textbf{Physics, a Beautiful Science} 
		\end{center}
		
	\end{questions}		
\end{document}